% Options for packages loaded elsewhere
\PassOptionsToPackage{unicode}{hyperref}
\PassOptionsToPackage{hyphens}{url}
\PassOptionsToPackage{dvipsnames,svgnames,x11names}{xcolor}
%
\documentclass[
  11pt,
  letterpaper,
  DIV=11,
  numbers=noendperiod]{scrartcl}

\usepackage{amsmath,amssymb}
\usepackage{setspace}
\usepackage{iftex}
\ifPDFTeX
  \usepackage[T1]{fontenc}
  \usepackage[utf8]{inputenc}
  \usepackage{textcomp} % provide euro and other symbols
\else % if luatex or xetex
  \usepackage{unicode-math}
  \defaultfontfeatures{Scale=MatchLowercase}
  \defaultfontfeatures[\rmfamily]{Ligatures=TeX,Scale=1}
\fi
\usepackage[]{lmodern}
\ifPDFTeX\else  
    % xetex/luatex font selection
\fi
% Use upquote if available, for straight quotes in verbatim environments
\IfFileExists{upquote.sty}{\usepackage{upquote}}{}
\IfFileExists{microtype.sty}{% use microtype if available
  \usepackage[]{microtype}
  \UseMicrotypeSet[protrusion]{basicmath} % disable protrusion for tt fonts
}{}
\makeatletter
\@ifundefined{KOMAClassName}{% if non-KOMA class
  \IfFileExists{parskip.sty}{%
    \usepackage{parskip}
  }{% else
    \setlength{\parindent}{0pt}
    \setlength{\parskip}{6pt plus 2pt minus 1pt}}
}{% if KOMA class
  \KOMAoptions{parskip=half}}
\makeatother
\usepackage{xcolor}
\usepackage[margin=1in,top=0.9in,bottom=0.9in]{geometry}
\setlength{\emergencystretch}{3em} % prevent overfull lines
\setcounter{secnumdepth}{5}
% Make \paragraph and \subparagraph free-standing
\makeatletter
\ifx\paragraph\undefined\else
  \let\oldparagraph\paragraph
  \renewcommand{\paragraph}{
    \@ifstar
      \xxxParagraphStar
      \xxxParagraphNoStar
  }
  \newcommand{\xxxParagraphStar}[1]{\oldparagraph*{#1}\mbox{}}
  \newcommand{\xxxParagraphNoStar}[1]{\oldparagraph{#1}\mbox{}}
\fi
\ifx\subparagraph\undefined\else
  \let\oldsubparagraph\subparagraph
  \renewcommand{\subparagraph}{
    \@ifstar
      \xxxSubParagraphStar
      \xxxSubParagraphNoStar
  }
  \newcommand{\xxxSubParagraphStar}[1]{\oldsubparagraph*{#1}\mbox{}}
  \newcommand{\xxxSubParagraphNoStar}[1]{\oldsubparagraph{#1}\mbox{}}
\fi
\makeatother


\providecommand{\tightlist}{%
  \setlength{\itemsep}{0pt}\setlength{\parskip}{0pt}}\usepackage{longtable,booktabs,array}
\usepackage{calc} % for calculating minipage widths
% Correct order of tables after \paragraph or \subparagraph
\usepackage{etoolbox}
\makeatletter
\patchcmd\longtable{\par}{\if@noskipsec\mbox{}\fi\par}{}{}
\makeatother
% Allow footnotes in longtable head/foot
\IfFileExists{footnotehyper.sty}{\usepackage{footnotehyper}}{\usepackage{footnote}}
\makesavenoteenv{longtable}
\usepackage{graphicx}
\makeatletter
\newsavebox\pandoc@box
\newcommand*\pandocbounded[1]{% scales image to fit in text height/width
  \sbox\pandoc@box{#1}%
  \Gscale@div\@tempa{\textheight}{\dimexpr\ht\pandoc@box+\dp\pandoc@box\relax}%
  \Gscale@div\@tempb{\linewidth}{\wd\pandoc@box}%
  \ifdim\@tempb\p@<\@tempa\p@\let\@tempa\@tempb\fi% select the smaller of both
  \ifdim\@tempa\p@<\p@\scalebox{\@tempa}{\usebox\pandoc@box}%
  \else\usebox{\pandoc@box}%
  \fi%
}
% Set default figure placement to htbp
\def\fps@figure{htbp}
\makeatother
% definitions for citeproc citations
\NewDocumentCommand\citeproctext{}{}
\NewDocumentCommand\citeproc{mm}{%
  \begingroup\def\citeproctext{#2}\cite{#1}\endgroup}
\makeatletter
 % allow citations to break across lines
 \let\@cite@ofmt\@firstofone
 % avoid brackets around text for \cite:
 \def\@biblabel#1{}
 \def\@cite#1#2{{#1\if@tempswa , #2\fi}}
\makeatother
\newlength{\cslhangindent}
\setlength{\cslhangindent}{1.5em}
\newlength{\csllabelwidth}
\setlength{\csllabelwidth}{3em}
\newenvironment{CSLReferences}[2] % #1 hanging-indent, #2 entry-spacing
 {\begin{list}{}{%
  \setlength{\itemindent}{0pt}
  \setlength{\leftmargin}{0pt}
  \setlength{\parsep}{0pt}
  % turn on hanging indent if param 1 is 1
  \ifodd #1
   \setlength{\leftmargin}{\cslhangindent}
   \setlength{\itemindent}{-1\cslhangindent}
  \fi
  % set entry spacing
  \setlength{\itemsep}{#2\baselineskip}}}
 {\end{list}}
\usepackage{calc}
\newcommand{\CSLBlock}[1]{\hfill\break\parbox[t]{\linewidth}{\strut\ignorespaces#1\strut}}
\newcommand{\CSLLeftMargin}[1]{\parbox[t]{\csllabelwidth}{\strut#1\strut}}
\newcommand{\CSLRightInline}[1]{\parbox[t]{\linewidth - \csllabelwidth}{\strut#1\strut}}
\newcommand{\CSLIndent}[1]{\hspace{\cslhangindent}#1}

\usepackage{amsmath}
\usepackage{amsthm}
\usepackage{amssymb}
\usepackage{amsfonts}
\usepackage{booktabs}
\usepackage{setspace}
\usepackage{graphicx}
\usepackage{algorithm}
\usepackage[noend]{algpseudocode}
% Define math operators if needed
\DeclareMathOperator*{\argmax}{arg\,max}
\DeclareMathOperator*{\Exp}{\mathbb{E}} % Expectation operator
\newcommand{\sym}[1]{\ensuremath{^{#1}}}
\KOMAoption{captions}{tableheading}
\makeatletter
\@ifpackageloaded{caption}{}{\usepackage{caption}}
\AtBeginDocument{%
\ifdefined\contentsname
  \renewcommand*\contentsname{Table of contents}
\else
  \newcommand\contentsname{Table of contents}
\fi
\ifdefined\listfigurename
  \renewcommand*\listfigurename{List of Figures}
\else
  \newcommand\listfigurename{List of Figures}
\fi
\ifdefined\listtablename
  \renewcommand*\listtablename{List of Tables}
\else
  \newcommand\listtablename{List of Tables}
\fi
\ifdefined\figurename
  \renewcommand*\figurename{Figure}
\else
  \newcommand\figurename{Figure}
\fi
\ifdefined\tablename
  \renewcommand*\tablename{Table}
\else
  \newcommand\tablename{Table}
\fi
}
\@ifpackageloaded{float}{}{\usepackage{float}}
\floatstyle{ruled}
\@ifundefined{c@chapter}{\newfloat{codelisting}{h}{lop}}{\newfloat{codelisting}{h}{lop}[chapter]}
\floatname{codelisting}{Listing}
\newcommand*\listoflistings{\listof{codelisting}{List of Listings}}
\makeatother
\makeatletter
\makeatother
\makeatletter
\@ifpackageloaded{caption}{}{\usepackage{caption}}
\@ifpackageloaded{subcaption}{}{\usepackage{subcaption}}
\makeatother

\usepackage{bookmark}

\IfFileExists{xurl.sty}{\usepackage{xurl}}{} % add URL line breaks if available
\urlstyle{same} % disable monospaced font for URLs
\hypersetup{
  pdftitle={The Value of Flexibility: Teleworkability, Sorting, and the Work‑From‑Home Wage Gap},
  pdfauthor={Mitchell Valdes-Bobes; Anna Lukianova},
  colorlinks=true,
  linkcolor={blue},
  filecolor={Maroon},
  citecolor={Blue},
  urlcolor={Blue},
  pdfcreator={LaTeX via pandoc}}


\title{The Value of Flexibility: Teleworkability, Sorting, and the
Work‑From‑Home Wage Gap}
\author{Mitchell Valdes-Bobes \and Anna Lukianova}
\date{July 31, 2025}

\begin{document}
\maketitle
\begin{abstract}
\begin{itemize}
\tightlist
\item[$\square$]
  Update abstract.
\end{itemize}
\end{abstract}


\setstretch{1.5}
\section{Introduction}\label{introduction}

\begin{itemize}
\tightlist
\item[$\square$]
  Update introduction.
\end{itemize}

\section{Empirical Motivation: Remote Work, Wages, and
Skills}\label{empirical-motivation-remote-work-wages-and-skills}

\begin{itemize}
\tightlist
\item[$\square$]
  Update empirical motivation.
\end{itemize}

\section{Model}\label{model}

This section develops a search and matching model of the labor market
with two-sided heterogeneity and endogenous remote work arrangements.
The model builds on the canonical
\textcolor{red}{ADD MAIN MODEL CITATION} framework, extended to include
heterogeneity in worker skills and firm capacity for remote work. This
framework allows us to analyze how sorting patterns, wage differentials,
and aggregate labor market outcomes are shaped by the trade-offs between
productivity and the non-pecuniary benefits of remote work.

\subsection{Environment}\label{environment}

The economy is populated by a continuum of infinitely lived workers and
a continuum of firms. Time is discrete and the horizon is infinite. Both
workers and firms are risk-neutral and discount the future at a common
rate \(\beta\).

\subsubsection{Agents}\label{agents}

\textbf{Workers} are indexed by their skill level, \(h \in \mathcal{H}\)
, which is exogenously distributed according to the cumulative
distribution function \(F_h(h)\)
\textcolor{red}{I NEED TO THINK ABOUT THIS MAYBE I DO UNIFORM ON THIS SIDE OF THE MARKET AND PRODUCTIVITY IS BASED ON RANK}.
Workers derive utility from their wage, \(w\), and the fraction of time
they work remotely, \(\alpha \in\). The instantaneous utility function
is given by \(u(w, \alpha)\). We assume utility is quasi-linear to
ensure that the choice of \(\alpha\) is independent of the wage
transfer, which simplifies the bargaining process. Specifically, we
adopt the functional form: \[u(w, \alpha) = w - c(1-\alpha)\] where
\(c(1-\alpha)\) represents the disutility from working in the office for
a fraction \(1-\alpha\) of the time.
\textcolor{red}{ADD PROPERTIES OF COST FUNCTION}

\textbf{Firms} are indexed by their remote work efficiency,
\(\psi \in \Psi\), which is exogenously distributed according to the
cumulative distribution function \(F_\psi(\psi)\). Firms produce a
single homogeneous good. The cost of posting \(v\) vacancies is given by
an increasing and convex function \(c(v)\).

\textbf{Production Technology:} output is generated by a match between a
worker of skill \(h\) and a firm with remote efficiency \(\psi\). The
production function depends on the share of remote work, \(\alpha\):
\[Y(\alpha \mid h, \psi) = A(h) \cdot \left[(1 - \alpha) + \alpha \cdot g(h, \psi)\right]\]
where \(A(h)\) is the baseline productivity of a worker with skill
\(h\), and \(g(h, \psi)\) is the productivity of remote work relative to
in-person work, which is normalized to one. We assume \(A'(h) > 0\), so
higher-skilled workers are more productive.

The relative productivity of remote work, \(g(h, \psi)\), is a key
feature of the model, capturing complementarities between worker skill
and firm technology. We assume \(g_\psi(h, \psi) \geq 0\) and
\(g_h(h, \psi) \geq 0\), meaning remote work is more effective in firms
with higher remote efficiency and for higher-skilled workers. The sign
of the cross-partial derivative, \(g_{h\psi}\), will determine the
nature of sorting in the market.

\subsubsection{Searching and Matching}\label{searching-and-matching}

The labor market is characterized by search frictions. Let \(u(h)\) be
the measure of unemployed workers of skill \(h\), and \(v(\psi)\) be the
number of vacancies posted by firms of type \(\psi\). The aggregate
measures of unemployed workers and vacancies are given respectively by:
\[L = \int u(h) dF_h(h) \quad  \text{and} \quad V = \int v(\psi) dF_\psi(\psi)\]
Meetings between unemployed workers and vacancies are governed by a
constant returns to scale matching function, \(M(L, V)\). The rate at
which an unemployed worker contacts a vacancy is
\(p(\theta) = M(L,V)/L\), and the rate at which a vacancy is filled is
\(q(\theta) = M(L,V)/V\), where \(\theta = V/L\) is the labor market
tightness. The probability that a searching worker meets a firm of a
specific type \(\psi\) is given by the proportion of vacancies of that
type, \(v(\psi)/V\).

\subsection{Value Functions and
Bargaining}\label{value-functions-and-bargaining}

We now define the value functions for workers and firms and describe the
bargaining process that determines wages and work arrangements.

\subsubsection{Value of Unemployment and
Vacancies}\label{value-of-unemployment-and-vacancies}

An unemployed worker of skill \(h\) receives unemployment benefits
\(b(h)\) and searches for a job. The value of being unemployed,
\(U(h)\), is described by the following Bellman equation:
\[U(h) = b(h) + \beta \mathbb{E} \left[ \max\{W(h, \psi), U(h)\} \right].\]
where \(W(h, \psi)\) is the value to a worker of being employed in a
match with a firm of type \(\psi\). The expectation is taken over the
distribution of vacancies the worker might encounter. Assuming workers
get a share \(\xi\) of the match surplus, \(S(h, \psi)\), through Nash
bargaining, we have: \[W(h, \psi) = U(h) + \xi S(h, \psi).\]The value of
unemployment can then be written as:
\[U(h) = b(h) + \beta \left[ U(h) + p(\theta) \xi \int S(h, \psi)^+ d\Gamma_v(\psi) \right]\]
where \(\Gamma_v(\psi)\) is the endogenous distribution of vacancies and
\(S(h, \psi)^+ = \max\{S(h, \psi), 0\}\). Solving for \(U(h)\) yields:
\[U(h) = \frac{b(h) + \beta p(\theta) \xi \int S(h, \psi)^+ d\Gamma_v(\psi)}{1 - \beta}\]
The value of a vacant position is zero due to a free-entry condition,
which will be detailed in the vacancy creation section.

\subsubsection{Value of a Match and Surplus
Sharing}\label{value-of-a-match-and-surplus-sharing}

A match between a worker \(h\) and a firm \(\psi\) generates a flow of
output and utility. Matches are subject to an exogenous destruction
shock with probability \(\delta\). We assume efficient bargaining: the
firm and worker first choose the remote work share \(\alpha\) to
maximize the joint value of the match, and then use the wage \(w\) to
divide the resulting surplus.

Given the quasi-linear utility, the total flow value generated by a
match is the sum of output net of the worker's disutility from in-office
work: \(Y(\alpha \mid h, \psi) - c(1-\alpha)\). The total surplus of the
match, \(S(h, \psi)\), is the difference between the value of the match,
\(J(h, \psi)\), and the worker's outside option, \(U(h)\). The surplus
evolves according to the Bellman equation:
\[S(h, \psi) = s(h, \psi) + \beta(1-\delta)S(h, \psi) - \beta p(\theta) \xi \int S(h, \psi')^+ d\gamma_v(\psi')\]
where \(s(h, \psi)\) is the flow surplus, defined as the maximized joint
value net of the worker's flow value of unemployment:
\[s(h, \psi) = \max_{\alpha \in} \left\{ Y(\alpha \mid h, \psi) - c(1-\alpha) \right\} - b(h)\]
Solving for the total match surplus gives:
\[S(h, \psi) = \frac{s(h, \psi) - \beta p(\theta) \xi \int S(h, \psi')^+ d\gamma_v(\psi')}{1 - \beta(1-\delta)}\]
This equation highlights that the value of a match depends not only on
its own flow surplus but also on the worker's expected future gains from
searching while unemployed.

\subsection{Equilibrium}\label{equilibrium}

An equilibrium in this economy consists of a set of value functions
(\(U(h)\), \(S(h, \psi)\)), policy functions for remote work
(\(\alpha^*(h, \psi)\)) and wages (\(w^*(h, \psi)\)), vacancy posting
decisions (\(v(\psi)\)), and a distribution of unemployed workers
(\(u(h)\)) and employed workers (\(n(h, \psi)\)), such that: 1. The
policy functions \(\alpha^*(h, \psi)\) and \(w^*(h, \psi)\) are
determined by efficient bargaining. 2. Firms' vacancy creation decisions
\(v(\psi)\) are optimal. 3. The labor market is in a steady state, where
the flows of workers into and out of unemployment are balanced.

\subsubsection{Optimal Work Arrangements and
Wages}\label{optimal-work-arrangements-and-wages}

The optimal remote work share, \(\alpha^*(h, \psi)\), is chosen to
maximize the flow surplus. The first-order condition for an interior
solution is \(Y'(\alpha) = c'(\alpha)\). This condition equates the
marginal gain in output from an additional unit of in-person work to the
marginal disutility for the worker. Analyzing this condition at the
boundaries \(\alpha=0\) and \(\alpha=1\) defines two skill-dependent
thresholds for firm efficiency, \(\underline{\psi}(h)\) and
\(\overline{\psi}(h)\), which partition the market into three regimes:
\[\alpha^*(h, \psi) = \begin{cases} 0 & \text{if } \psi \leq \underline{\psi}(h) \quad \text{(Full In-Person)} \\ \alpha_{\text{interior}}(h, \psi) & \text{if } \underline{\psi}(h) < \psi < \overline{\psi}(h) \quad \text{(Hybrid)} \\ 1 & \text{if } \psi \geq \overline{\psi}(h) \quad \text{(Full Remote)} \end{cases}\]

Under the assumption of efficient bargaining, the remote work share,
\(\alpha^*(h, \psi)\), is chosen to maximize the total joint value of
the match. The wage, \(w\), is then determined as a purely
distributional transfer to ensure that the total surplus,
\(S(h, \psi)\), is divided according to the exogenously given bargaining
powers, \(\xi\) for the worker and \(1-\xi\) for the firm.

The wage is set to deliver the worker their promised lifetime value from
the match, \(W(h, \psi)\). This value is composed of their outside
option, the value of unemployment \(U(h)\), plus their share of the
total match surplus.

With the optimal physical arrangement \(\alpha^*(h, \psi)\) determined,
we can solve for the wage \(w^*(h, \psi)\) that supports this
equilibrium. The wage must be set such that the worker's value from the
match equals their outside option plus their bargained share of the
surplus. This implies that the wage must deliver the required utility
flow and compensate for any in-office disutility. \[
w^*(h, \psi) = \underbrace{\left( W(h, \psi) - \beta \mathbb{E}[W'] \right)}_{\text{Required utility flow}} \quad + \underbrace{c(\alpha^*(h, \psi))}_{\text{Compensation for in-office work}}
\]

\subsubsection{Vacancy Creation}\label{vacancy-creation}

Firms post vacancies up to the point where the marginal cost of posting
equals the expected marginal benefit. The expected benefit is the
probability of filling a vacancy, \(q(\theta)\), multiplied by the
firm's share of the expected surplus from a new match. The free-entry
condition is therefore:
\[c'(v(\psi)) = q(\theta) (1-\xi) \int S(h, \psi)^+ \frac{u(h)}{L} dF_h(h)\]
This condition determines the number of vacancies \(v(\psi)\) for each
firm type, which in turn determines the aggregate market tightness
\(\theta\) and the distributions of job-finding and job-filling
probabilities.

\subsubsection{Steady-State Flows}\label{steady-state-flows}

In a steady-state equilibrium, the mass of workers of each type and in
each state (employed or unemployed) is constant. This requires that the
flow of workers out of unemployment equals the flow into unemployment.
The flow out of unemployment for workers of skill \(h\) is:
\[p(\theta) u(h) \int \mathbb{1}_{\{S(h, \psi) > 0\}} d\Gamma_v(\psi).\]It
is the product of the mass of unemployed workers of skill \(h\) , and
their effective job-finding rate, which is the probability of meeting
any firm \(p(\theta)\) , multiplied by the probability that a meeting
results in a mutually agreeable match (i.e., one with positive surplus,
\(S(h, \psi)>0\)). The flow into unemployment is the sum of all job
destructions: \[\delta \int n(h, \psi) dF_\psi(\psi).\] The steady-state
condition (Beveridge curve) is given by equating the aggregate flows:
\[\delta \iint n(h, \psi) dF_\psi(\psi) dF_h(h) = p(\theta) \int u(h) \left( \int \mathbb{1}_{\{S(h, \psi) > 0\}} d\gamma_v(\psi) \right) dF_h(h)\]
The equilibrium is a fixed point where firms' vacancy decisions are
optimal given expected surpluses, and these surpluses are consistent
with the hiring probabilities that result from those same vacancy
decisions.

\newpage

\section{Calibration}\label{calibration}

\newpage

\section{Results}\label{results}

\section{Conclusion}\label{conclusion}

\begin{itemize}
\tightlist
\item[$\square$]
  Update conclusion. This is an example reference: (Aksoy et al. 2023)
\end{itemize}

\section*{References}\label{references}
\addcontentsline{toc}{section}{References}

\phantomsection\label{refs}
\begin{CSLReferences}{1}{0}
\bibitem[\citeproctext]{ref-aksoy2023}
Aksoy, Cevat Giray, Jose Maria Barrero, Nicholas Bloom, Steven Davis,
Mathias Dolls, and Pablo Zarate. 2023. {``Time {Savings When Working}
from {Home}.''} Workign Paper. \url{https://doi.org/10.3386/w30866}.

\end{CSLReferences}

\newpage

\setcounter{section}{0}
\renewcommand{\thesection}{\Alph{section}}

\setcounter{table}{0}
\renewcommand{\thetable}{A\arabic{table}}

\setcounter{figure}{0}
\renewcommand{\thefigure}{A\arabic{figure}}

\section{Appendix}\label{appendix}




\end{document}
