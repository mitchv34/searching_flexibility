% Options for packages loaded elsewhere
\PassOptionsToPackage{unicode}{hyperref}
\PassOptionsToPackage{hyphens}{url}
\PassOptionsToPackage{dvipsnames,svgnames,x11names}{xcolor}
%
\documentclass[
  11pt,
  letterpaper,
  DIV=11,
  numbers=noendperiod]{scrartcl}

\usepackage{amsmath,amssymb}
\usepackage{setspace}
\usepackage{iftex}
\ifPDFTeX
  \usepackage[T1]{fontenc}
  \usepackage[utf8]{inputenc}
  \usepackage{textcomp} % provide euro and other symbols
\else % if luatex or xetex
  \usepackage{unicode-math}
  \defaultfontfeatures{Scale=MatchLowercase}
  \defaultfontfeatures[\rmfamily]{Ligatures=TeX,Scale=1}
\fi
\usepackage[]{lmodern}
\ifPDFTeX\else  
    % xetex/luatex font selection
\fi
% Use upquote if available, for straight quotes in verbatim environments
\IfFileExists{upquote.sty}{\usepackage{upquote}}{}
\IfFileExists{microtype.sty}{% use microtype if available
  \usepackage[]{microtype}
  \UseMicrotypeSet[protrusion]{basicmath} % disable protrusion for tt fonts
}{}
\makeatletter
\@ifundefined{KOMAClassName}{% if non-KOMA class
  \IfFileExists{parskip.sty}{%
    \usepackage{parskip}
  }{% else
    \setlength{\parindent}{0pt}
    \setlength{\parskip}{6pt plus 2pt minus 1pt}}
}{% if KOMA class
  \KOMAoptions{parskip=half}}
\makeatother
\usepackage{xcolor}
\usepackage[margin=1in,top=0.9in,bottom=0.9in]{geometry}
\setlength{\emergencystretch}{3em} % prevent overfull lines
\setcounter{secnumdepth}{5}
% Make \paragraph and \subparagraph free-standing
\makeatletter
\ifx\paragraph\undefined\else
  \let\oldparagraph\paragraph
  \renewcommand{\paragraph}{
    \@ifstar
      \xxxParagraphStar
      \xxxParagraphNoStar
  }
  \newcommand{\xxxParagraphStar}[1]{\oldparagraph*{#1}\mbox{}}
  \newcommand{\xxxParagraphNoStar}[1]{\oldparagraph{#1}\mbox{}}
\fi
\ifx\subparagraph\undefined\else
  \let\oldsubparagraph\subparagraph
  \renewcommand{\subparagraph}{
    \@ifstar
      \xxxSubParagraphStar
      \xxxSubParagraphNoStar
  }
  \newcommand{\xxxSubParagraphStar}[1]{\oldsubparagraph*{#1}\mbox{}}
  \newcommand{\xxxSubParagraphNoStar}[1]{\oldsubparagraph{#1}\mbox{}}
\fi
\makeatother


\providecommand{\tightlist}{%
  \setlength{\itemsep}{0pt}\setlength{\parskip}{0pt}}\usepackage{longtable,booktabs,array}
\usepackage{calc} % for calculating minipage widths
% Correct order of tables after \paragraph or \subparagraph
\usepackage{etoolbox}
\makeatletter
\patchcmd\longtable{\par}{\if@noskipsec\mbox{}\fi\par}{}{}
\makeatother
% Allow footnotes in longtable head/foot
\IfFileExists{footnotehyper.sty}{\usepackage{footnotehyper}}{\usepackage{footnote}}
\makesavenoteenv{longtable}
\usepackage{graphicx}
\makeatletter
\newsavebox\pandoc@box
\newcommand*\pandocbounded[1]{% scales image to fit in text height/width
  \sbox\pandoc@box{#1}%
  \Gscale@div\@tempa{\textheight}{\dimexpr\ht\pandoc@box+\dp\pandoc@box\relax}%
  \Gscale@div\@tempb{\linewidth}{\wd\pandoc@box}%
  \ifdim\@tempb\p@<\@tempa\p@\let\@tempa\@tempb\fi% select the smaller of both
  \ifdim\@tempa\p@<\p@\scalebox{\@tempa}{\usebox\pandoc@box}%
  \else\usebox{\pandoc@box}%
  \fi%
}
% Set default figure placement to htbp
\def\fps@figure{htbp}
\makeatother
% definitions for citeproc citations
\NewDocumentCommand\citeproctext{}{}
\NewDocumentCommand\citeproc{mm}{%
  \begingroup\def\citeproctext{#2}\cite{#1}\endgroup}
\makeatletter
 % allow citations to break across lines
 \let\@cite@ofmt\@firstofone
 % avoid brackets around text for \cite:
 \def\@biblabel#1{}
 \def\@cite#1#2{{#1\if@tempswa , #2\fi}}
\makeatother
\newlength{\cslhangindent}
\setlength{\cslhangindent}{1.5em}
\newlength{\csllabelwidth}
\setlength{\csllabelwidth}{3em}
\newenvironment{CSLReferences}[2] % #1 hanging-indent, #2 entry-spacing
 {\begin{list}{}{%
  \setlength{\itemindent}{0pt}
  \setlength{\leftmargin}{0pt}
  \setlength{\parsep}{0pt}
  % turn on hanging indent if param 1 is 1
  \ifodd #1
   \setlength{\leftmargin}{\cslhangindent}
   \setlength{\itemindent}{-1\cslhangindent}
  \fi
  % set entry spacing
  \setlength{\itemsep}{#2\baselineskip}}}
 {\end{list}}
\usepackage{calc}
\newcommand{\CSLBlock}[1]{\hfill\break\parbox[t]{\linewidth}{\strut\ignorespaces#1\strut}}
\newcommand{\CSLLeftMargin}[1]{\parbox[t]{\csllabelwidth}{\strut#1\strut}}
\newcommand{\CSLRightInline}[1]{\parbox[t]{\linewidth - \csllabelwidth}{\strut#1\strut}}
\newcommand{\CSLIndent}[1]{\hspace{\cslhangindent}#1}

\usepackage{amsmath}
\usepackage{amsthm}
\usepackage{amssymb}
\usepackage{amsfonts}
\usepackage{booktabs}
\usepackage{setspace}
\usepackage{graphicx}
\usepackage{algorithm}
\usepackage[noend]{algpseudocode}
% Define math operators if needed
\DeclareMathOperator*{\argmax}{arg\,max}
\DeclareMathOperator*{\Exp}{\mathbb{E}} % Expectation operator
\newcommand{\sym}[1]{\ensuremath{^{#1}}}
\KOMAoption{captions}{tableheading}
\makeatletter
\@ifpackageloaded{caption}{}{\usepackage{caption}}
\AtBeginDocument{%
\ifdefined\contentsname
  \renewcommand*\contentsname{Table of contents}
\else
  \newcommand\contentsname{Table of contents}
\fi
\ifdefined\listfigurename
  \renewcommand*\listfigurename{List of Figures}
\else
  \newcommand\listfigurename{List of Figures}
\fi
\ifdefined\listtablename
  \renewcommand*\listtablename{List of Tables}
\else
  \newcommand\listtablename{List of Tables}
\fi
\ifdefined\figurename
  \renewcommand*\figurename{Figure}
\else
  \newcommand\figurename{Figure}
\fi
\ifdefined\tablename
  \renewcommand*\tablename{Table}
\else
  \newcommand\tablename{Table}
\fi
}
\@ifpackageloaded{float}{}{\usepackage{float}}
\floatstyle{ruled}
\@ifundefined{c@chapter}{\newfloat{codelisting}{h}{lop}}{\newfloat{codelisting}{h}{lop}[chapter]}
\floatname{codelisting}{Listing}
\newcommand*\listoflistings{\listof{codelisting}{List of Listings}}
\makeatother
\makeatletter
\makeatother
\makeatletter
\@ifpackageloaded{caption}{}{\usepackage{caption}}
\@ifpackageloaded{subcaption}{}{\usepackage{subcaption}}
\makeatother

\usepackage{bookmark}

\IfFileExists{xurl.sty}{\usepackage{xurl}}{} % add URL line breaks if available
\urlstyle{same} % disable monospaced font for URLs
\hypersetup{
  pdftitle={The Value of Flexibility: Teleworkability, Sorting, and the Work‑From‑Home Wage Gap},
  pdfauthor={Mitchell Valdes-Bobes; Anna Lukianova},
  colorlinks=true,
  linkcolor={blue},
  filecolor={Maroon},
  citecolor={Blue},
  urlcolor={Blue},
  pdfcreator={LaTeX via pandoc}}


\title{The Value of Flexibility: Teleworkability, Sorting, and the
Work‑From‑Home Wage Gap}
\author{Mitchell Valdes-Bobes \and Anna Lukianova}
\date{July 31, 2025}

\begin{document}
\maketitle
\begin{abstract}
\begin{itemize}
\tightlist
\item[$\square$]
  Update abstract.
\end{itemize}
\end{abstract}


\setstretch{1.5}
\section{Introduction}\label{introduction}

\begin{itemize}
\tightlist
\item[$\square$]
  Update introduction.
\end{itemize}

\section{Empirical Motivation: Remote Work, Wages, and
Skills}\label{empirical-motivation-remote-work-wages-and-skills}

\begin{itemize}
\tightlist
\item[$\square$]
  Update empirical motivation.
\end{itemize}

\section{Model}\label{model}

This section develops a search and matching model of the labor market
with two-sided heterogeneity and endogenous remote work arrangements.
The model builds on the canonical
\textcolor{red}{ADD MAIN MODEL CITATION} framework, extended to include
heterogeneity in worker skills and firm capacity for remote work. This
framework allows us to analyze how sorting patterns, wage differentials,
and aggregate labor market outcomes are shaped by the trade-offs between
productivity and the non-pecuniary benefits of remote work.

\subsection{Environment}\label{environment}

The economy is populated by a continuum of infinitely lived workers and
a continuum of firms. Time is discrete and the horizon is infinite. Both
workers and firms are risk-neutral and discount the future at a common
rate \(\beta\).

\subsubsection{Agents}\label{agents}

\textbf{Workers} are indexed by their skill level, \(h \in \mathcal{H}\)
, which is exogenously distributed according to the cumulative
distribution function \(F_h(h)\)
\textcolor{red}{I NEED TO THINK ABOUT THIS MAYBE I DO UNIFORM ON THIS SIDE OF THE MARKET AND PRODUCTIVITY IS BASED ON RANK}.
Workers derive utility from their wage, \(w\), and the fraction of time
they work remotely, \(\alpha \in\). The instantaneous utility function
is given by \(u(w, \alpha)\). We assume utility is quasi-linear to
ensure that the choice of \(\alpha\) is independent of the wage
transfer, which simplifies the bargaining process. Specifically, we
adopt the functional form: \[u(w, \alpha) = w - c(1-\alpha)\] where
\(c(1-\alpha)\) represents the disutility from working in the office for
a fraction \(1-\alpha\) of the time.
\textcolor{red}{ADD PROPERTIES OF COST FUNCTION}

\textbf{Firms} are indexed by their remote work efficiency,
\(\psi \in \Psi\), which is exogenously distributed according to the
cumulative distribution function \(F_\psi(\psi)\). Firms produce a
single homogeneous good. The cost of posting \(v\) vacancies is given by
an increasing and convex function \(c(v)\).

\textbf{Production Technology:} output is generated by a match between a
worker of skill \(h\) and a firm with remote efficiency \(\psi\). The
production function depends on the share of remote work, \(\alpha\):
\[Y(\alpha \mid h, \psi) = A(h) \cdot \left[(1 - \alpha) + \alpha \cdot g(h, \psi)\right]\]
where \(A(h)\) is the baseline productivity of a worker with skill
\(h\), and \(g(h, \psi)\) is the productivity of remote work relative to
in-person work, which is normalized to one. We assume \(A'(h) > 0\), so
higher-skilled workers are more productive.

The relative productivity of remote work, \(g(h, \psi)\), is a key
feature of the model, capturing complementarities between worker skill
and firm technology. We assume \(g_\psi(h, \psi) \geq 0\) and
\(g_h(h, \psi) \geq 0\), meaning remote work is more effective in firms
with higher remote efficiency and for higher-skilled workers. The sign
of the cross-partial derivative, \(g_{h\psi}\), will determine the
nature of sorting in the market.

\subsubsection{Searching and Matching}\label{searching-and-matching}

The labor market is characterized by search frictions. Let \(u(h)\) be
the measure of unemployed workers of skill \(h\), and \(v(\psi)\) be the
number of vacancies posted by firms of type \(\psi\). The aggregate
measures of unemployed workers and vacancies are given respectively by:
\[L = \int u(h) dF_h(h) \quad  \text{and} \quad V = \int v(\psi) dF_\psi(\psi)\]
Meetings between unemployed workers and vacancies are governed by a
constant returns to scale matching function, \(M(L, V)\). The rate at
which an unemployed worker contacts a vacancy is
\(p(\theta) = M(L,V)/L\), and the rate at which a vacancy is filled is
\(q(\theta) = M(L,V)/V\), where \(\theta = V/L\) is the labor market
tightness. The probability that a searching worker meets a firm of a
specific type \(\psi\) is given by the proportion of vacancies of that
type, \(v(\psi)/V\).

\subsection{Value Functions and
Bargaining}\label{value-functions-and-bargaining}

We now define the value functions for workers and firms and describe the
bargaining process that determines wages and work arrangements.

\subsubsection{Value of Unemployment and
Vacancies}\label{value-of-unemployment-and-vacancies}

An unemployed worker of skill \(h\) receives unemployment benefits
\(b(h)\) and searches for a job. The value of being unemployed,
\(U(h)\), is described by the following Bellman equation:
\[U(h) = b(h) + \beta \mathbb{E} \left[ \max\{W(h, \psi), U(h)\} \right].\]
where \(W(h, \psi)\) is the value to a worker of being employed in a
match with a firm of type \(\psi\). The expectation is taken over the
distribution of vacancies the worker might encounter. Assuming workers
get a share \(\xi\) of the match surplus, \(S(h, \psi)\), through Nash
bargaining, we have: \[W(h, \psi) = U(h) + \xi S(h, \psi).\]The value of
unemployment can then be written as:
\[U(h) = b(h) + \beta \left[ U(h) + p(\theta) \xi \int S(h, \psi)^+ d\Gamma_v(\psi) \right]\]
where \(\Gamma_v(\psi)\) is the endogenous distribution of vacancies and
\(S(h, \psi)^+ = \max\{S(h, \psi), 0\}\). Solving for \(U(h)\) yields:
\[U(h) = \frac{b(h) + \beta p(\theta) \xi \int S(h, \psi)^+ d\Gamma_v(\psi)}{1 - \beta}\]
The value of a vacant position is zero due to a free-entry condition,
which will be detailed in the vacancy creation section.

\subsubsection{Value of a Match and Surplus
Sharing}\label{value-of-a-match-and-surplus-sharing}

A match between a worker \(h\) and a firm \(\psi\) generates a flow of
output and utility. Matches are subject to an exogenous destruction
shock with probability \(\delta\). We assume efficient bargaining: the
firm and worker first choose the remote work share \(\alpha\) to
maximize the joint value of the match, and then use the wage \(w\) to
divide the resulting surplus.

Given the quasi-linear utility, the total flow value generated by a
match is the sum of output net of the worker's disutility from in-office
work: \(Y(\alpha \mid h, \psi) - c(1-\alpha)\). The total surplus of the
match, \(S(h, \psi)\), is the difference between the value of the match,
\(J(h, \psi)\), and the worker's outside option, \(U(h)\). The surplus
evolves according to the Bellman equation:
\[S(h, \psi) = s(h, \psi) + \beta(1-\delta)S(h, \psi) - \beta p(\theta) \xi \int S(h, \psi')^+ d\gamma_v(\psi')\]
where \(s(h, \psi)\) is the flow surplus, defined as the maximized joint
value net of the worker's flow value of unemployment:
\[s(h, \psi) = \max_{\alpha \in} \left\{ Y(\alpha \mid h, \psi) - c(1-\alpha) \right\} - b(h)\]
Solving for the total match surplus gives:
\[S(h, \psi) = \frac{s(h, \psi) - \beta p(\theta) \xi \int S(h, \psi')^+ d\gamma_v(\psi')}{1 - \beta(1-\delta)}\]
This equation highlights that the value of a match depends not only on
its own flow surplus but also on the worker's expected future gains from
searching while unemployed.

\subsection{Equilibrium}\label{equilibrium}

An equilibrium in this economy consists of a set of value functions
(\(U(h)\), \(S(h, \psi)\)), policy functions for remote work
(\(\alpha^*(h, \psi)\)) and wages (\(w^*(h, \psi)\)), vacancy posting
decisions (\(v(\psi)\)), and a distribution of unemployed workers
(\(u(h)\)) and employed workers (\(n(h, \psi)\)), such that: 1. The
policy functions \(\alpha^*(h, \psi)\) and \(w^*(h, \psi)\) are
determined by efficient bargaining. 2. Firms' vacancy creation decisions
\(v(\psi)\) are optimal. 3. The labor market is in a steady state, where
the flows of workers into and out of unemployment are balanced.

\subsubsection{Optimal Work Arrangements and
Wages}\label{optimal-work-arrangements-and-wages}

The optimal remote work share, \(\alpha^*(h, \psi)\), is chosen to
maximize the flow surplus. The first-order condition for an interior
solution is \(Y'(\alpha) = c'(\alpha)\). This condition equates the
marginal gain in output from an additional unit of in-person work to the
marginal disutility for the worker. Analyzing this condition at the
boundaries \(\alpha=0\) and \(\alpha=1\) defines two skill-dependent
thresholds for firm efficiency, \(\underline{\psi}(h)\) and
\(\overline{\psi}(h)\), which partition the market into three regimes:
\[\alpha^*(h, \psi) = \begin{cases} 0 & \text{if } \psi \leq \underline{\psi}(h) \quad \text{(Full In-Person)} \\ \alpha_{\text{interior}}(h, \psi) & \text{if } \underline{\psi}(h) < \psi < \overline{\psi}(h) \quad \text{(Hybrid)} \\ 1 & \text{if } \psi \geq \overline{\psi}(h) \quad \text{(Full Remote)} \end{cases}\]

Under the assumption of efficient bargaining, the remote work share,
\(\alpha^*(h, \psi)\), is chosen to maximize the total joint value of
the match. The wage, \(w\), is then determined as a purely
distributional transfer to ensure that the total surplus,
\(S(h, \psi)\), is divided according to the exogenously given bargaining
powers, \(\xi\) for the worker and \(1-\xi\) for the firm.

The wage is set to deliver the worker their promised lifetime value from
the match, \(W(h, \psi)\). This value is composed of their outside
option, the value of unemployment \(U(h)\), plus their share of the
total match surplus.

With the optimal physical arrangement \(\alpha^*(h, \psi)\) determined,
we can solve for the wage \(w^*(h, \psi)\) that supports this
equilibrium. The wage must be set such that the worker's value from the
match equals their outside option plus their bargained share of the
surplus. This implies that the wage must deliver the required utility
flow and compensate for any in-office disutility. \[
w^*(h, \psi) = \underbrace{\left( W(h, \psi) - \beta \mathbb{E}[W'] \right)}_{\text{Required utility flow}} \quad + \underbrace{c(\alpha^*(h, \psi))}_{\text{Compensation for in-office work}}
\]

\subsubsection{Vacancy Creation}\label{vacancy-creation}

Firms post vacancies up to the point where the marginal cost of posting
equals the expected marginal benefit. The expected benefit is the
probability of filling a vacancy, \(q(\theta)\), multiplied by the
firm's share of the expected surplus from a new match. The free-entry
condition is therefore:
\[c'(v(\psi)) = q(\theta) (1-\xi) \int S(h, \psi)^+ \frac{u(h)}{L} dF_h(h)\]
This condition determines the number of vacancies \(v(\psi)\) for each
firm type, which in turn determines the aggregate market tightness
\(\theta\) and the distributions of job-finding and job-filling
probabilities.

\subsubsection{Steady-State Flows}\label{steady-state-flows}

In a steady-state equilibrium, the mass of workers of each type and in
each state (employed or unemployed) is constant. This requires that the
flow of workers out of unemployment equals the flow into unemployment.
The flow out of unemployment for workers of skill \(h\) is:
\[p(\theta) u(h) \int \mathbb{1}_{\{S(h, \psi) > 0\}} d\Gamma_v(\psi).\]It
is the product of the mass of unemployed workers of skill \(h\) , and
their effective job-finding rate, which is the probability of meeting
any firm \(p(\theta)\) , multiplied by the probability that a meeting
results in a mutually agreeable match (i.e., one with positive surplus,
\(S(h, \psi)>0\)). The flow into unemployment is the sum of all job
destructions: \[\delta \int n(h, \psi) dF_\psi(\psi).\] The steady-state
condition (Beveridge curve) is given by equating the aggregate flows:
\[\delta \iint n(h, \psi) dF_\psi(\psi) dF_h(h) = p(\theta) \int u(h) \left( \int \mathbb{1}_{\{S(h, \psi) > 0\}} d\gamma_v(\psi) \right) dF_h(h)\]
The equilibrium is a fixed point where firms' vacancy decisions are
optimal given expected surpluses, and these surpluses are consistent
with the hiring probabilities that result from those same vacancy
decisions.

\section{Calibration}\label{calibration}

\subsection{Quantitative Specification and
Parameterization}\label{quantitative-specification-and-parameterization}

To take the model to the data and analyze its quantitative implications,
we must specify functional forms for the model's key components and
assign numerical values to the resulting parameters. We divide the
parameters into two groups: those set externally based on standard
values in the literature, and those calibrated internally to match key
moments of the U.S. labor market data.

\subsubsection{Functional Forms}\label{functional-forms}

\textcolor{red}{NEED TO SETTLE ON ONE FUNCTIONAL FORM FOR THE $g(h, \psi)$ FUNCTION}

\textbf{Production and Cost Functions:} We adopt specific functional
forms for the production and worker cost functions to derive an
analytical solution for the optimal work arrangement, \(\alpha^*\).

\begin{itemize}
\tightlist
\item
  \textbf{In-Office Cost Function:} The worker's disutility from
  in-office work is assumed to be a convex function of the time spent in
  the office, \((1-\alpha)\).
  \[c(1-\alpha) = c_0 \frac{(1-\alpha)^{1+\chi}}{1+\chi}\] The parameter
  \(c_0 > 0\) scales the overall disutility, and the curvature parameter
  \(\chi > 0\) ensures that the marginal disutility of in-office time is
  increasing, a standard feature that helps ensure well-behaved interior
  solutions for \(\alpha\).
\end{itemize}

\textbf{Production Function:} The baseline productivity, \(A(h)\), is
assumed to be linear in worker skill: \[A(h) = A_1 h\]The relative
productivity of remote work, \(g(h, \psi)\), is specified using a power
function to capture the interaction between worker and firm
characteristics in determining remote output:
\[g(h, \psi) = \psi_0 h^\phi \psi^\nu\]Here, \(A_1\) is a productivity
scalar. In the remote productivity function, \(\psi_0\) is a scaling
factor, while \(\phi\) and \(\nu\) are the elasticities of remote
productivity with respect to worker skill (\(h\)) and firm efficiency
(\(\psi\)), respectively. The parameter \(\phi\) is particularly
important as it governs whether skill and firm technology are
complements (\(\phi > 0\)) or substitutes (\(\phi < 0\)) in remote
production.

These functional forms allow us to analytically characterize the
thresholds that govern the optimal work arrangement. The upper
threshold, \(\overline{\psi}(h)\), which separates hybrid from
full-remote work, is strictly decreasing in worker skill whenever skill
and technology are complements (\(\phi > 0\)). This provides the
intuitive result that higher-skilled workers require a lower level of
firm efficiency to make full remote work optimal.

The shape of the lower threshold, \(\underline{\psi}(h)\), is richer, as
it is determined by a tension between the rising opportunity cost of
remote work (the loss of a more productive worker from the office) and
the skill-remote interaction effect. The nature of this trade-off
depends critically on the skill-remote elasticity, \(\phi\):

\begin{itemize}
\item
  \textbf{When skill and technology are complements (\(\phi > 0\)):} The
  lower threshold has an \textbf{inverted U-shape}. For low-skill
  workers, the rising opportunity cost dominates, making the threshold
  increase with skill. For high-skill workers, the powerful
  complementarity effect dominates, making the threshold decrease with
  skill. This implies that middle-skill workers face the highest
  efficiency requirement to be offered a hybrid arrangement.
\item
  \textbf{When skill and technology are weak substitutes
  (\(-1 < \phi < 0\)):} The lower threshold is \textbf{strictly
  increasing}. As a worker's skill increases, they become both more
  valuable in the office (higher opportunity cost) and relatively less
  effective remotely (due to substitutability). Both forces push in the
  same direction, requiring a progressively higher level of firm
  efficiency to justify a hybrid offer.
\item
  \textbf{When skill and technology are strong substitutes
  (\(\phi < -1\)):} The lower threshold has a \textbf{U-shape}. For
  low-skill workers, the strong substitutability effect is dominant,
  causing the threshold to fall with skill. For high-skill workers, the
  opportunity cost effect eventually dominates again, causing the
  threshold to rise. This implies that middle-skill workers are the
  \emph{most} likely to receive a hybrid offer.
\end{itemize}

This rich set of possibilities highlights the flexibility of the
framework in capturing diverse sorting patterns. The full mathematical
derivation of these cases and the formulas are presented in
Section~\ref{sec-appendix-g-cd}

\textbf{Matching and Vacancy Costs:}

\begin{itemize}
\tightlist
\item
  We assume a standard Cobb-Douglas matching function, which exhibits
  constant returns to scale. \[
  M(L, V) = \gamma_{0} L^{\gamma_{1}} V^{1-\gamma_{1}}
  \] where \(\gamma_0 > 0\) is matching efficiency and
  \(\gamma_1 \in (0,1)\) is the elasticity of matches with respect to
  unemployment.
\end{itemize}

\subsubsection{Firm's Vacancy Posting
Problem}\label{firms-vacancy-posting-problem}

A firm of type \(\psi\) chooses the number of vacancies \(v\) to post to
maximize its expected profit. The profit function is the expected
benefit minus the total cost: \[
\Pi(v(\psi)) = q B(\psi)  v - c(v)
\] where: * \(B(\psi) = (1-\xi)\mathbb{E}_h[S(h,\psi)^{+}]\) is the
firm's expected benefit \emph{per filled vacancy}. * \textbf{Vacancy
Cost Function:} The cost of posting vacancies is assumed to be a convex
function, implying a rising marginal cost to posting.
\[c(v) = \frac{\kappa_{0} v^{1+\kappa_{1}}}{1+\kappa_{1}}\] where
\(\kappa_{0} > 0\) is a cost scaling parameter and \(\kappa_{1} > 0\)
ensures convexity.

The firm's first-order condition (FOC) for profit maximization is
\(\Pi'(v) = 0\), which implies that the marginal benefit must equal the
marginal cost: \[
\Pi'(v) = 0 \quad \implies \quad q B(\psi) = c'(v) \quad \implies \quad q B(\psi) = \kappa_{0} v(\psi)^{\kappa_{1}}
\] This condition allows us to derive an equilibrium condition for
\(\theta\). Details of this derivation are provided in
Section~\ref{sec-appendix-cobb-douglas}.

\subsubsection{Calibration}\label{calibration-1}

We choose the parameter values to align the model's steady state with
key features of the pre-pandemic U.S. labor market. The model is
calibrated at a monthly frequency. The parameters are summarized in
Table 1.

\textbf{Externally Set Parameters:} Several parameters are set to
standard values from the literature. The discount factor \(\beta\) is
set to 0.996, corresponding to a 5\% annual real interest rate. The
elasticity of the matching function, \(\eta\), is set to 0.5, a common
value in the search and matching literature (Petrongolo and Pissarides,
2001).

\textbf{Internally Calibrated Parameters:} The remaining parameters are
calibrated jointly to match a set of moments from the data.

\textcolor{red}{EXAMPLE TEXT} ``The parameters of the production
function (\(A_1, \nu, \psi_0, \phi\)) and the in-office cost function
(\(c_0, \chi\)) are chosen to match the average labor productivity, the
observed share of workers in remote/hybrid arrangements, and the
estimated compensating wage differential for remote work from survey
evidence (e.g., Barrero et al., 2023a). The vacancy cost parameters
(\(c_v, \gamma\)) and matching efficiency (\(M_0\)) are disciplined by
the average labor market tightness (\(\theta\)), the job-filling rate
(\(q\)), and the elasticity of vacancies to productivity
shocks\ldots{}''

\textcolor{red}{EXAMPLE TABLE}

\begin{table}[h!]
\centering
\caption{Parameter Values}

\begin{tabular}{@{}llcl@{}}
\toprule
\textbf{Parameter} & \textbf{Description} & \textbf{Value} & \textbf{Target/Source} \\
\midrule
\multicolumn{4}{l}{\textit{Preferences \& Technology}} \\
$\beta$ & Discount Factor & 0.996 & 5\% annual interest rate \\
$\chi$ & Curvature of in-office cost & 1.5 & Internally calibrated \\
$c_0$ & In-office cost scale & 0.25 & Match compensating differential \\
$A_1$ & Productivity scale & 1.0 & Normalization \\
$\phi$ & Skill-remote complementarity & 0.1 & Match skill premium \\
... & ... & ... & ... \\
\addlinespace % Adds a bit of vertical space for separation
\multicolumn{4}{l}{\textit{Search \& Matching}} \\
$\eta$ & Matching elasticity & 0.5 & Petrongolo \& Pissarides (2001) \\
$\delta$ & Exogenous separation rate & 0.025 & Match avg. job duration \\
$\xi$ & Worker bargaining power & 0.5 & Shimer (2005) \\
... & ... & ... & ... \\
\bottomrule
\end{tabular}
\end{table}

\newpage

\section{Results}\label{results}

\section{Conclusion}\label{conclusion}

\begin{itemize}
\tightlist
\item[$\square$]
  Update conclusion. This is an example reference: (Aksoy et al. 2023)
\end{itemize}

\section*{References}\label{references}
\addcontentsline{toc}{section}{References}

\phantomsection\label{refs}
\begin{CSLReferences}{1}{0}
\bibitem[\citeproctext]{ref-aksoy2023}
Aksoy, Cevat Giray, Jose Maria Barrero, Nicholas Bloom, Steven Davis,
Mathias Dolls, and Pablo Zarate. 2023. {``Time {Savings When Working}
from {Home}.''} Workign Paper. \url{https://doi.org/10.3386/w30866}.

\end{CSLReferences}

\newpage

\setcounter{section}{0}
\renewcommand{\thesection}{\Alph{section}}

\setcounter{table}{0}
\renewcommand{\thetable}{A\arabic{table}}

\setcounter{figure}{0}
\renewcommand{\thefigure}{A\arabic{figure}}

\section{Appendix}\label{appendix}

\subsection{Derivation of the Optimal Work
Arrangement}\label{sec-appendix-g-cd}

This appendix derives the optimal remote work policy,
\(\alpha^*(h, \psi)\), and analyzes the properties of the threshold
functions that partition the labor market into distinct work arrangement
regimes.

\subsubsection{The Maximization Problem}\label{the-maximization-problem}

Under the assumption of efficient bargaining, the firm and worker choose
the remote work share \(\alpha \in\) to maximize the total joint flow
value of the match. With quasi-linear utility, this simplifies to
maximizing the sum of output \(Y(\alpha)\) minus the worker's in-office
disutility \(c(1-\alpha)\): \[
\max_{\alpha \in [0,1]} \quad \Big\{Y(\alpha \mid h, \psi) - c(1-\alpha)\Big\}
\] To solve this problem, we use the functional forms specified in the
main text:

\begin{itemize}
\tightlist
\item
  \textbf{Production Function:}
  \(Y(\alpha \mid h, \psi) = A(h) \cdot \left[(1 - \alpha) + \alpha \cdot g(h, \psi)\right]\)
\item
  \textbf{Baseline Productivity:} \(A(h) = A_1 h\)
\item
  \textbf{Relative Remote Productivity:}
  \(g(h, \psi) = \psi_0 h^{\phi} \psi^{\nu}\)
\item
  \textbf{In-Office Cost Function:}
  \(c(1-\alpha) = c_0 \frac{(1-\alpha)^{1+\chi}}{1+\chi}\)
\end{itemize}

The First-Order Condition (FOC) is found by differentiating the joint
surplus \(S(\alpha)\) with respect to \(\alpha\) and setting it to zero.

\begin{itemize}
\tightlist
\item
  \textbf{Derivative of Output:}
  \(\frac{dY}{d\alpha} = A_1 h (g(h, \psi) - 1)\)
\item
  \textbf{Derivative of Cost:}
  \(\frac{d}{d\alpha} \left( -c(1-\alpha) \right) = - \left( c_0(1-\alpha)^\chi \cdot (-1) \right) = c_0(1-\alpha)^\chi\)
\end{itemize}

The FOC is therefore: \[
A_1 h (g(h, \psi) - 1) + c_0(1-\alpha)^{\chi} = 0
\] Rearranging this gives the fundamental equation for an interior
solution: \[
A_1 h (1 - g(h, \psi)) = c_0(1-\alpha)^{\chi}
\] For an interior solution to exist, the left-hand side must be
positive, which requires \(1 - g(h, \psi) > 0\), or \(g(h, \psi) < 1\).
This means remote work must be less productive than in-person work for a
hybrid arrangement to be optimal.

\subsubsection{Optimal Policy and
Thresholds}\label{optimal-policy-and-thresholds}

\textbf{Interior Solution}

Solving the FOC for the interior solution
\(\alpha_{\text{interior}} \in (0,1)\): \[
\alpha_{\text{interior}}(h, \psi) = 1 - \left[ \frac{A_1 h \left( 1 - g(h, \psi) \right)}{c_0} \right]^{\frac{1}{\chi}} = 1 - \left[ \frac{A_1 h \left( 1 - \psi_0 h^\phi \psi^\nu \right)}{c_0} \right]^{\frac{1}{\chi}}
\]

\textbf{Threshold Derivations}

The boundaries of the hybrid region are determined by analyzing the
derivative of the surplus, \(\frac{d(Y-c)}{d\alpha}\), at the corners
\(\alpha=0\) and \(\alpha=1\).

\begin{enumerate}
\def\labelenumi{\arabic{enumi}.}
\item
  \textbf{Upper Threshold, \(\overline{\psi}(h)\):} The boundary for
  full remote work (\(\alpha^*=1\)) is where

  \begin{align*}
  A_1 h (g(h, \overline{\psi}(h)) - 1) + c_0(1-1)^\chi &= 0 \\
  A_1 h (g(h, \overline{\psi}(h)) - 1) &= 0 \\
  \implies \overline{\psi}(h) &= \psi_0^{-1/\nu} h^{-\phi/\nu}
  \end{align*}
\item
  \textbf{Lower Threshold, \(\underline{\psi}(h)\):} The boundary for
  full in-person work (\(\alpha^*=0\)) is where

  \begin{align*}
  A_1 h (g(h, \underline{\psi}(h)) - 1) + c_0(1-0)^\chi &= 0 \\
  A_1 h (g(h, \underline{\psi}(h)) - 1) &= -c_0 \\
  g(h, \underline{\psi}(h)) &= 1 - \frac{c_0}{A_1 h} \\
  \psi_0 h^\phi [\underline{\psi}(h)]^\nu &= 1 - \frac{c_0}{A_1 h} \\
  \implies \underline{\psi}(h) &= \left( \frac{1}{\psi_0 h^\phi} \left( 1 - \frac{c_0}{A_1 h} \right) \right)^{1/\nu}\\
  \implies \underline{\psi}(h) & = \overline{\psi}(h)\ \left( 1 - \frac{c_0}{A_1 h} \right)^{1/\nu}
  \end{align*}
\end{enumerate}

\subsubsection{Monotonicity and
Properties}\label{monotonicity-and-properties}

\textbf{Monotonicity of the Thresholds}

\begin{itemize}
\item
  \textbf{Upper Threshold \(\overline{\psi}(h)\):} \[
  \frac{\partial \overline{\psi}(h)}{\partial h} = \psi_0^{-1/\nu} \left(-\frac{\phi}{\nu}\right) h^{-\frac{\phi}{\nu}-1}
  \] Assuming skill-remote complementarity (\(\phi > 0\)), the upper
  threshold is \textbf{strictly decreasing} in worker skill.
  Higher-skilled workers require a lower level of firm efficiency to
  make full remote work optimal.
\item
  \textbf{Lower Threshold \(\underline{\psi}(h)\):} The shape of this
  corrected threshold is now more complex, let
  \(f(h) = \frac{1}{\psi_0} \left( h^{-\phi} - \frac{c_0}{A_1}h^{-\phi-1} \right)\),
  so that \(\underline{\psi}(h) = [f(h)]^{1/\nu}\). The monotonicity is
  determined by the sign of \(f'(h)\): \[
  f'(h) = \frac{1}{\psi_0} \left( -\phi h^{-\phi-1} + \frac{c_0(\phi+1)}{A_1}h^{-\phi-2} \right) = \frac{h^{-\phi-2}}{\psi_0} \left( -\phi h + \frac{c_0(\phi+1)}{A_1} \right)
  \] Setting the term in the parenthesis to zero gives the critical
  point \(\hat{h}\): \[
  \hat{h} = \frac{c_0}{A_1} \frac{\phi+1}{\phi} = \frac{c_0}{A_1} \left(1 + \frac{1}{\phi}\right)
  \] The shape of the threshold now depends critically on the value of
  the skill-remote elasticity \(\phi\). Consider the term: \[
  T(h) = -\phi h + \frac{c_0(\phi+1)}{A_1}
  \] The sign of \(\frac{\partial\underline{\psi}(h)}{\partial h}\) is
  the same as the sign of \(T(h)\). This term captures the tension
  between two economic forces:

  \begin{enumerate}
  \def\labelenumi{\arabic{enumi}.}
  \tightlist
  \item
    \textbf{The Skill-Remote Interaction Effect (\(-\phi h\)):} This
    term reflects how a worker's skill directly alters their
    productivity in a remote setting. Its effect depends critically on
    the sign of \(\phi\).
  \item
    \textbf{The Opportunity Cost Effect (\(\frac{c_0(\phi+1)}{A_1}\)):}
    This term reflects the cost of forgoing a worker's full baseline
    productivity (\(A_1 h\)) by having them work remotely. The firm must
    be compensated for this loss by the worker's increased effectiveness
    or willingness to accept a lower wage (captured by \(c_0\)).
  \end{enumerate}

  Before analyzing the cases, let's define our key terms based on the
  cross-partial derivative of the remote productivity function,
  \(g(h, \psi) = \psi_0 h^\phi \psi^\nu\): \[
  \frac{\partial^2 g}{\partial h \partial \psi} = \psi_0 \phi \nu h^{\phi-1} \psi^{\nu-1}
  \] The sign of this expression is determined entirely by the sign of
  \(\phi\).

  \begin{itemize}
  \tightlist
  \item
    \textbf{Complementarity (\(\phi > 0\)):} Worker skill and firm
    remote-efficiency are complements. An increase in firm efficiency
    (\(\psi\)) raises the marginal productivity of worker skill (\(h\))
    in remote work, and vice-versa.
  \item
    \textbf{Substitutability (\(\phi < 0\)):} Worker skill and firm
    remote-efficiency are substitutes. An increase in firm efficiency
    \emph{lowers} the marginal productivity of worker skill in remote
    work. This can be thought of as a situation where the firm's
    technology is so effective that it makes the worker's innate skill
    less relevant for remote tasks.
  \end{itemize}
\item
  \textbf{Case 1: Strong Complementarity (\(\phi > 0\)):} The lower
  threshold \(\underline{\psi}(h)\) has an \textbf{inverted U-shape},
  peaking at \(\hat{h} = \frac{c_0}{A_1}(1 + \frac{1}{\phi})\).

  \begin{itemize}
  \tightlist
  \item
    \textbf{Economic Intuition:} This is the most intuitive case. When
    skill and firm technology are complements, the threshold's shape is
    driven by a trade-off that changes with the skill level.

    \begin{itemize}
    \tightlist
    \item
      \textbf{For Low-Skill Workers (\(h < \hat{h}\)):} At low skill
      levels, the complementarity effect (\(h^\phi\)) is weak and the
      worker's baseline productivity (\(A_1 h\)) is low. As skill \(h\)
      increases from a low base, the \textbf{opportunity cost effect}
      dominates. The loss of baseline productivity from not being in the
      office grows faster than the gain from the weak complementarity.
      To justify offering even a small amount of remote work, the firm
      requires progressively higher remote efficiency (\(\psi\)).
      Therefore, the threshold \(\underline{\psi}(h)\) is
      \textbf{increasing}.
    \item
      \textbf{For High-Skill Workers (\(h > \hat{h}\)):} At high skill
      levels, the \textbf{skill-remote interaction effect} becomes
      dominant. The complementarity is now powerful; a small increase in
      \(h\) makes the worker significantly more effective with the
      firm's remote technology. This strong gain in remote productivity
      now outweighs the linear increase in opportunity cost. The firm is
      willing to offer a hybrid arrangement even with a lower level of
      its own remote efficiency (\(\psi\)) because the worker's high
      skill compensates for it. Therefore, the threshold
      \(\underline{\psi}(h)\) is \textbf{decreasing}.
    \end{itemize}
  \end{itemize}
\item
  \textbf{Case 2: Weak Substitutability (\(-1 < \phi < 0\)):} The lower
  threshold \(\underline{\psi}(h)\) is \textbf{strictly increasing}.

  \begin{itemize}
  \tightlist
  \item
    \textbf{Economic Intuition:} In this regime, worker skill and firm
    technology are substitutes. An increase in worker skill has two
    negative consequences for the firm's incentive to offer remote work:

    \begin{enumerate}
    \def\labelenumi{\arabic{enumi}.}
    \tightlist
    \item
      The \textbf{opportunity cost} of not having the worker in the
      office (\(A_1 h\)) increases, making in-person work more valuable.
    \item
      The \textbf{relative remote productivity} (\(h^\phi\)) actually
      \emph{decreases} as \(h\) rises, because \(\phi\) is negative. The
      worker becomes comparatively worse at remote work as their skill
      increases.
    \end{enumerate}
  \item
    Both economic forces push in the same direction. As a worker's skill
    increases, they become simultaneously more valuable in the office
    and less effective remotely. To overcome this ``double penalty'' and
    still find it optimal to offer a hybrid arrangement, the firm must
    possess a substantially higher level of its own remote efficiency
    (\(\psi\)). Consequently, the minimum required efficiency,
    \(\underline{\psi}(h)\), must be \textbf{strictly increasing} with
    worker skill.
  \end{itemize}
\end{itemize}

\textbf{Case 3: Strong Substitutability (\(\phi < -1\))} The lower
threshold \(\underline{\psi}(h)\) is \textbf{U-shaped}, with a minimum
at \(\hat{h} = \frac{c_0}{A_1}(1 + \frac{1}{\phi})\).

\begin{itemize}
\tightlist
\item
  \textbf{Economic Intuition:} This is the most complex case. While
  skill and technology are still substitutes, the relationship is so
  strong that it reverses the pattern seen in \textbf{Case 2}.

  \begin{itemize}
  \tightlist
  \item
    \textbf{For Low-Skill Workers (\(h < \hat{h}\)):} At very low skill
    levels, the \textbf{strong substitutability} dominates. An increase
    in skill \(h\) makes the worker so much relatively worse at remote
    work (the \(h^\phi\) term with \(\phi < -1\) falls very rapidly)
    that this effect outweighs the rising opportunity cost. To
    compensate for this sharp decline in relative remote fitness, the
    firm needs less of its own efficiency (\(\psi\)) to be indifferent,
    as the trade-off is already heavily skewed toward in-person work.
    The threshold \(\underline{\psi}(h)\) is \textbf{decreasing}.
  \item
    \textbf{For High-Skill Workers (\(h > \hat{h}\)):} As skill becomes
    sufficiently high, the logic from Case 2 takes over. The
    \textbf{opportunity cost effect} begins to dominate again. The loss
    of a highly productive worker from the office becomes the primary
    concern for the firm. Even though the worker is a substitute for
    technology, their high baseline productivity makes keeping them in
    the office very attractive. To entice the firm to offer a hybrid
    arrangement, the firm's own remote efficiency (\(\psi\)) must be
    increasingly high. Therefore, the threshold \(\underline{\psi}(h)\)
    is \textbf{increasing}. Of course. This is an excellent next step,
    as the size of the hybrid region is a key outcome of the model that
    determines how many worker-firm pairs can even consider a non-corner
    solution.
  \end{itemize}
\end{itemize}

\subsubsection{\texorpdfstring{\textbf{Size and Properties of the Hybrid
Region}}{Size and Properties of the Hybrid Region}}\label{size-and-properties-of-the-hybrid-region}

The existence of a hybrid work arrangement is possible for a worker of
skill \(h\) only if there is a non-empty range of firm efficiencies
\(\psi\) such that \(\underline{\psi}(h) < \psi < \overline{\psi}(h)\).
We define the size of this hybrid region as the width of this interval:
\[
\Delta\psi(h) = \overline{\psi}(h) - \underline{\psi}(h) = \left( \psi_0^{-1/\nu} h^{-\phi/\nu} \right) \left[ 1 - \left( 1 - \frac{c_0}{A_1 h} \right)^{1/\nu} \right]
\] This region is well-defined for all \(h\) such that
\(1 - \frac{c_0}{A_1 h} > 0\), which requires \(h > c_0/A_1\). For skill
levels below this minimum, no remote work is ever optimal.

\textbf{Analysis of the Hybrid Region's Size}

To understand how the range of hybrid opportunities changes with worker
skill, we analyze the derivative \(\frac{d(\Delta\psi(h))}{dh}\). The
behavior depends critically on the nature of the skill-remote
interaction, governed by \(\phi\).

\begin{itemize}
\tightlist
\item
  \textbf{Case 1: Strong Complementarity (\(\phi > 0\)):} The size of
  the hybrid region, \(\Delta\psi(h)\), is \textbf{strictly decreasing}
  in worker skill \(h\).

  \begin{itemize}
  \tightlist
  \item
    \textbf{Economic Intuition:} When skill and firm technology are
    complements, higher-skilled workers are pushed towards the
    full-remote corner solution more rapidly than they are pulled away
    from the full-in-person corner.
  \end{itemize}

  \begin{enumerate}
  \def\labelenumi{\arabic{enumi}.}
  \tightlist
  \item
    \textbf{Upper Threshold Effect:} The upper threshold
    \(\overline{\psi}(h)\) is decreasing in \(h\). As a worker's skill
    increases, their high remote productivity means they require a
    progressively lower level of firm efficiency to make full remote
    work optimal. This effect shrinks the hybrid region from above.
  \item
    \textbf{Threshold Gap Effect:} The term in the brackets,
    \(\left[ 1 - \left( 1 - \frac{c_0}{A_1 h} \right)^{1/\nu} \right]\),
    represents the gap between the thresholds as a fraction of the upper
    threshold. This gap also shrinks as \(h\) increases. Intuitively, as
    a worker's baseline productivity \(A_1 h\) rises, the cost of being
    in the office (\(c_0\)) becomes smaller in relative terms, meaning
    the lower threshold moves closer to the upper threshold.
  \end{enumerate}
\item
  \textbf{Case 2: Substitutability (\(\phi < 0\)):} The monotonicity of
  the hybrid region's size, \(\Delta\psi(h)\), is \textbf{ambiguous} and
  depends on the specific parameter values.

  \begin{itemize}
  \tightlist
  \item
    \textbf{Economic Intuition:} When skill and technology are
    substitutes, two competing economic forces are at play:

    \begin{enumerate}
    \def\labelenumi{\arabic{enumi}.}
    \tightlist
    \item
      \textbf{Upper Threshold Effect (Widening):} The upper threshold
      \(\overline{\psi}(h)\) is now \emph{increasing} in \(h\). As a
      worker's skill increases, their relative remote productivity
      (\(h^\phi\)) falls. This makes it \emph{harder} for them to
      qualify for the full-remote regime, which pushes the upper
      boundary outwards and tends to \textbf{widen} the hybrid region.
    \item
      \textbf{Threshold Gap Effect (Narrowing):} The relative gap
      between the thresholds continues to shrink as \(h\) increases, for
      the same reason as in the complementarity case (the relative
      importance of \(c_0\) diminishes). This effect tends to
      \textbf{narrow} the hybrid region.
    \end{enumerate}
  \end{itemize}
\end{itemize}

\subsection{Market Tightness Equilibrium
Derivation}\label{sec-appendix-cobb-douglas}

We assume a standard Cobb-Douglas matching function, which exhibits
constant returns to scale. \[
M(L, V) = \gamma_{0} L^{\gamma_{1}} V^{1-\gamma_{1}}
\] This function yields the following probabilities:

\begin{itemize}
\tightlist
\item
  \textbf{Job-Filling Rate:}
  \(q(\theta) = \gamma_{0} \theta^{-\gamma_{1}}\)
\item
  \textbf{Job-Finding Rate:}
  \(p(\theta) = \gamma_{0} \theta^{1-\gamma_{1}}\)
\end{itemize}

\textbf{Optimal Vacancy Posting}

We begin with the firm's FOC and solve for the optimal number of
vacancies \(v(\psi)\): \[
q B(\psi) = \kappa_{0} v(\psi)^{\kappa_{1}} \implies v(\psi) = \left( \frac{B(\psi)}{\kappa_{0}} q \right)^{1/\kappa_{1}}
\] Substituting the Cobb-Douglas job-filling rate,
\(q(\theta) = \gamma_{0} \theta^{-\gamma_{1}}\), gives the expression
for \(v(\psi)\) as a function of market tightness \(\theta\) and the
firm's expected benefit \(B(\psi)\): \[
v(\psi) = \left( \frac{B(\psi)}{\kappa_{0}} \frac{\gamma_{0}}{\theta^{\gamma_{1}}} \right)^{1/\kappa_{1}}
\]

\textbf{Equilibrium Market Tightness}

The aggregate number of vacancies, \(V\), is found by integrating
\(v(\psi)\) across all firm types: \[
V = \int v(\psi) dF_{\psi}(\psi) = \int \left( \frac{\gamma_0 B(\psi)}{\kappa_0 \theta^{\gamma_1}} \right)^{1/\kappa_1} dF_{\psi}(\psi)
\] We can factor out the terms that do not depend on the firm type
\(\psi\): \[
V = \left( \frac{\gamma_0}{\kappa_0 \theta^{\gamma_1}} \right)^{1/\kappa_1} \int [B(\psi)]^{1/\kappa_1} dF_{\psi}(\psi) = \frac{1}{\theta^{\gamma_1/\kappa_1}} \left( \frac{\gamma_0}{\kappa_0} \right)^{1/\kappa_1} \int [B(\psi)]^{1/\kappa_1} dF_{\psi}(\psi)
\] To find the equilibrium, we substitute the definition of market
tightness, \(V = \theta L\), into the left-hand side: \[
\theta L = \frac{1}{\theta^{\gamma_1/\kappa_1}} \left( \frac{\gamma_0}{\kappa_0} \right)^{1/\kappa_1} \int [B(\psi)]^{1/\kappa_1} dF_{\psi}(\psi)
\] Now, we solve for \(\theta\) by grouping all \(\theta\) terms on the
left: \[
\theta^{1 + \gamma_1/\kappa_1} L = \left( \frac{\gamma_0}{\kappa_0} \right)^{1/\kappa_1} \int [B(\psi)]^{1/\kappa_1} dF_{\psi}(\psi)
\] Finally, isolating \(\theta\) yields the closed-form solution for
equilibrium market tightness: \[
\theta = \left( \frac{1}{L} \left( \frac{\gamma_0}{\kappa_0} \right)^{1/\kappa_1} \int [B(\psi)]^{1/\kappa_1} dF_{\psi}(\psi) \right)^{\frac{\kappa_1}{\kappa_1 + \gamma_1}}
\]




\end{document}
